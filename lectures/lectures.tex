\documentclass{scrartcl}
\usepackage[utf8]{inputenc}
\usepackage[T2A]{fontenc}
\usepackage[russian]{babel}
\usepackage{amssymb}
\usepackage{amsmath}
\usepackage{listings}
\newtheorem{theorem}{Теорема}
\newtheorem{task}{Задача}
\newtheorem{corollary}{Следствие}[theorem]
\newtheorem{lemma}[theorem]{Лемма}
\title{Лекции по дискретной математике.}
\author{Титилин Александр}
\date{}
\begin{document}
\maketitle
\section{Вычислительная геометрия.}
\begin{task}
    Есть два вектора $p_1 p_2$ c началом в точке $(0,0)$. Найти направление поврота вектора  $p_1$ по отношению к $p_2$ 
\end{task}
\[
\vec{p_1} = (x_1,y_1)
.\] 
\[
\vec{p_2} = (x_2,y_2)
.\] 
Ищем их векторное произведение
\[
    \begin{vmatrix}
        x_1 & x_2\\
        y_1 & y_2
    \end{vmatrix}  = x_1y_2 - x_2y_1
.\] 
Знак показывает направление (отрицаельный по часовой, иначе против)
Если определитель равен нулю, то векторы коллинеарны
\begin{task}
    Ломанная $\overline{p_1p_2p_3}$ составлена из двух отрезков $\overline{p_1p_2}$ и $\overline{p_2p_3}$. В каком напревление осуществляется поворот
    при переходе через точку $p_2$?
\end{task}
Cчитаем векторное произведение $\vec{p_1p_2}$ и $\vec{p_1p_3}$
Дальше как в прошлой задаче
\begin{task}
	Найти выпуклую оболочку заданную множества точек. (Выпуклый многоугольник внутри которого все точки)
\end{task}
\begin{task}
    Два отрезка заданы координатами своих концов. Пересекаются ли они?
\end{task}
Cначала для каждого отрезка определим ограничивающий прямоугольник (прямоугольник, у которого данный отрезок является диагональю)
Если ограничиввающие прямоугольники не пересекаются, то и отрезки не пересекаются.
Если пересекаются, то проверяем лежатли $p_1 p_2$ по разные стороны от прямой $p_3p_4$, и лежат ли $p_3 p_4$ по разные стороны $p_1p_2$. $p_3 p_4$ лежат по разные стороны от $p_1p_2$, если 
$\vec{p_1p_3}$ и $\vec{p_1p_4}$ имеют различную ориентацию относительно $\vec{p_1p_2}$
\subsection{Поиск пересекающихся отрезков}
Есть множество отрезков, где нет вертикальных и никакие три не пересекаются в одной точке. Надо найти пару пересекающихся отрезков.

С отрезком мы можем делать следущие операции
\begin{enumerate}
    \item Определить пересекается ли с другим
    \item
        Определитель, какой из двух заданных непересекающихся отрезков данный момент выше относительно другого.
\end{enumerate}
Проводим вертикальную прямую, двигаем ее слева направо и отмечаем изменение порядка ординат точек пересекающих прямую
В начале работы все концы отрезков упорядочиваем по абциссами,  а 
для одинаковых абцисс по ординатам.
В момент добавления отрезка проверяем, не пересекается ли он с соседними.

При удалении проверяем не пересекаются ли соседние отрезки

Алгоритм находит пару пересекающихся отрезков за $O(n \log{n}$
\subsection{Алгоритм Грэхема}
\begin{enumerate}
	\item Сначала найдем самую нижную точку (если таких несколько, то самую левую из них)
	\item Все оставшиеся точки отсортируем по возрастанию полярного угла
	\item Рассматриваем точки по очереди, начиная с четвертой. Если поворот от предыдущей точки происходил направо, то предыдущую точку исключаем и повторяем проверку.
\end{enumerate}
Алгоритм находит выпуклую оболочку за $O(n \log{n})$
\subsection{Алгоритм Джарвиса}
\begin{enumerate}
    \item тоже смое что в прошлом алгоритме
    \item На каждом следущем шаге, выбираем из
        еще не вошедших в оболочку, точку, лежащую в направлении, имеющем минимальный положительный yгол по сравнению с предыдущим направлениемю
\end{enumerate}
АЛгоритм джарвиса становится лучше Грэхема, если количество точек оболочки меньше $\log{n}$
\end{document}
