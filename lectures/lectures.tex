\documentclass{scrartcl}
\usepackage[utf8]{inputenc}
\usepackage[T2A]{fontenc}
\usepackage[russian]{babel}
\usepackage{amssymb}
\usepackage{amsmath}
\usepackage{listings}
\newtheorem{theorem}{Теорема}
\newtheorem{task}{Задача}
\newtheorem{corollary}{Следствие}[theorem]
\newtheorem{lemma}[theorem]{Лемма}
\title{Лекции по дискретной математике.}
\author{Титилин Александр}
\date{}
\begin{document}
\maketitle
\section{Вычислительная геометрия.}
Есть два вектора из начала координат, $A = (x_1,y_1) , B  = (x_2, y_2)$.
Их векторное произведение $\begin{vmatrix}
		x_1 & x_2 & i \\
		y_1 & y_2 & j \\
		0   & 0   & k
	\end{vmatrix} $
\begin{task}
	В какую сторону повернуть вектор $p_1$ на минимальный угол, чтоб он совпал по направлению с $p_2$
\end{task}
Вычисляем векторное произведение, определяем его знак. Знак определяет направление поворота. Если определитель ноль
\begin{task}
	Ломанная $\overline{p_1p_2p_3}$ составлена их двух отрезков $\overline{p_1p_2}$
	и $\overline{p_2 p_3}$. В каком напревдении осуществляется поворот при переходе через точку $p_2$
\end{task}
\begin{task}
	Два отрезка заданы координатами своих концов. Пересекаются ли они?
\end{task}
Определяем для каждого из отрезков ограничивающий прямоугольник. Если они не пересекаются то и отрезки не пересекаются. Продолжаем отрезки для прямых и точки концов кадого отрзка должны лежать по обе стороны прямой.
\begin{task}
	Есть множество отрезков, пересекаются ли хотя бы два???
\end{task}
Будем считать, то нет вертикальных и нет трех отрезков, которые пересекаются в одной точке.
Мы будем двигать вертикальную прямую слева направо, отмечаем моменты когда меняется ситуация (она начинает пересекать новый отрезок или перставать пересекать). После перехода, через точку пересечения отрезков ровно две ординаты меняются местами.
\begin{task}
	Найти выпуклую оболочку заданную множества точек. (Выпуклый многоугольник внутри которого все точки)
\end{task}
\subsection{Алгоритм Грэхема}
\begin{enumerate}
	\item Сначала найдем самую нижную точку (если таких несколько, то самую левую из них)
	\item Все оставшиеся точки отсортируем по возрастанию полярного угла
	\item Рассматриваем точки по очереди, начиная с четвертой. Если поворот от предыдущей точки происходил направо, то предыдущую точку исключаем и повторяем проверку.
\end{enumerate}
Алгоритм находит выпуклую оболочку за $O(n \log{n})$
\subsection{Алгоритм Джарвиса}
\begin{enumerate}
    \item тоже смое что в прошлом алгоритме
    \item На каждом следущем шаге, выбираем из
        еще не вошедших в оболочку, точку, лежащую в направлении, имеющем минимальный положительный yгол по сравнению с предыдущим направлениемю
\end{enumerate}
АЛгоритм джарвиса становится лучше Грэхема, если количество точек оболочки меньше $\log{n}$
\end{document}
