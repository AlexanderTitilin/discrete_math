\documentclass{scrartcl}
\usepackage[utf8]{inputenc}
\usepackage[T2A]{fontenc}
\usepackage[russian]{babel}
\usepackage{amssymb}
\usepackage{amsmath}
\usepackage{listings}
\newtheorem{theorem}{Теорема}
\newtheorem{task}{Задача}
\newtheorem{corollary}{Следствие}[theorem]
\newtheorem{lemma}[theorem]{Лемма}
\title{Лекции по дискретной математике.}
\author{Титилин Александр}
\date{}
\begin{document}
\maketitle
\section{Вычислительная геометрия.}
\begin{task}
    Есть два вектора $p_1 p_2$ c началом в точке $(0,0)$. Найти направление поврота вектора  $p_1$ по отношению к $p_2$ 
\end{task}
\[
\vec{p_1} = (x_1,y_1)
.\] 
\[
\vec{p_2} = (x_2,y_2)
.\] 
Ищем их векторное произведение
\[
    \begin{vmatrix}
        x_1 & x_2\\
        y_1 & y_2
    \end{vmatrix}  = x_1y_2 - x_2y_1
.\] 
Знак показывает направление (отрицаельный по часовой, иначе против)
Если определитель равен нулю, то векторы коллинеарны
\begin{task}
    Ломанная $\overline{p_1p_2p_3}$ составлена из двух отрезков $\overline{p_1p_2}$ и $\overline{p_2p_3}$. В каком напревление осуществляется поворот
    при переходе через точку $p_2$?
\end{task}
Cчитаем векторное произведение $\vec{p_1p_2}$ и $\vec{p_1p_3}$
Дальше как в прошлой задаче
\begin{task}
	Найти выпуклую оболочку заданную множества точек. (Выпуклый многоугольник внутри которого все точки)
\end{task}
\subsection{Алгоритм Грэхема}
\begin{enumerate}
	\item Сначала найдем самую нижную точку (если таких несколько, то самую левую из них)
	\item Все оставшиеся точки отсортируем по возрастанию полярного угла
	\item Рассматриваем точки по очереди, начиная с четвертой. Если поворот от предыдущей точки происходил направо, то предыдущую точку исключаем и повторяем проверку.
\end{enumerate}
Алгоритм находит выпуклую оболочку за $O(n \log{n})$
\subsection{Алгоритм Джарвиса}
\begin{enumerate}
    \item тоже смое что в прошлом алгоритме
    \item На каждом следущем шаге, выбираем из
        еще не вошедших в оболочку, точку, лежащую в направлении, имеющем минимальный положительный yгол по сравнению с предыдущим направлениемю
\end{enumerate}
АЛгоритм джарвиса становится лучше Грэхема, если количество точек оболочки меньше $\log{n}$
\end{document}
